\documentclass{article}
\usepackage{ctex}
\usepackage{amsmath}

\begin{document}
    \section{简介}
    \LaTeX{}将排版内容分为文本模式和数学模式。文本模式用于普通文本排版,数学模式用于数学公式排版。
    \section{行内公式}
    \subsection{美元符号}
    交换律是$a+b=b+a$,如$1+2=2+1=3$ 
    \subsection{小括号}
    交换律是\(a+b=b+a\),如\(1+2=2+1=3\) 
    \subsection{math环境}
    交换律是 \begin{math}a+b=b+a\end{math},如 \begin{math}1+2=2+1=3\end{math} 
    \section{上下标}
    \subsection{上标}
    $3x^2-x+2=0$,$3x^{20}-x+2=0$
    \subsection{下标}
    $a_0, a_1, a_2$

    $a_0, a_2, a_3, ..., a_{3x^2-x+2}$
    \section{希腊字母}
    $\alpha$
    $\beta$
    $\gamma$
    $\epsilon$
    $\pi$
    $\omega$

    $\Gamma$
    $\Delta$
    $\Theta$
    $\Pi$
    $\Omega$
    \section{数学函数}
    $\log$
    $\sin$
    $\cos$
    $\arcsin$
    $\arccos$
    $\arctan$
    $\ln$
    
    $\sin^2x+\cos^2x=1$

    $y=\arcsin x$

    $y=\sin^{-1}x$

    $y=\log_2x$

    $y=\ln x$

    $\sqrt{2}$
    $\sqrt{x^2 + y^2}$
    $\sqrt{2+\sqrt{2}}$
    $\sqrt[4]{x}$
    \section{分式}
    大约是原体积的3/4.

    大约是原体积的$\frac{3}{4}$
    \section{行间公式}
    \subsection{美元符号}
    交换律是$$a+b=b+a$$如$$1+2=2+1=3$$
    \subsection{中括号}
    交换律是\[a+b=b+a\]如\[1+2=2+1=3\]
    \subsection{displaymath环境}
    交换律是\begin{displaymath}a+b=b+a\end{displaymath}如\begin{displaymath}1+2=2+1=3\end{displaymath}
    \subsection{自动编号公式equation环境}
    交换律公式见式(\ref{equ1}).
    \begin{equation}
        a+b=b+a\label{equ1}
    \end{equation}
    \subsection{不编号公式equation*环境}
    交换律公式见式(\ref{equ2}).
    \begin{equation*}
        a+b=b+a\label{equ2}
    \end{equation*}

\end{document}