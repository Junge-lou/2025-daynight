\newcommand{\MyCMD}{\textbf{这是我们自定义的命令}}
\newcommand{\MyCMDone}[1]{\textbf{#1}}
\newcommand{\MyCMDtwo}[3]{(#1+#2)^#3}
\newcommand{\MyCMDthree}[3][x]{(#1+#2)^#3}

\chapter{命令}
\section{为什么定义命令}
对于命令,大家之前也了解过,一个  \textbackslash 加上相应的单词,就可以让编译器进行相应的操作,这就是命令,但是如果我们想对某一种类似的操作进行重复使用,那么直接进行各种命令的叠加会非常麻烦,所以我们需要定义自己命令来进行各种操作的集成,这就是我们为什么要进行定义新命令的原因。
\section{定义一个新命令}
\LaTeX{}的命令相当于其他编程语言中的函数,我们定义新命令的过程跟定义函数的过程也类似,首先我们来定义一个最简单的命令
\begin{lstlisting}[language=TeX]
\newcommand{\MyCMD}{\textbf{这是我们自定义的命令}}
\end{lstlisting}

其中,\textbackslash newcommand是标识符,告诉编译器我们要定义一个新命令,然后\textbackslash MyCMD是我们定义的新命令的标识符,类似于函数名,我们在文档中定义这个标识符之后,就可以进行使用,然后第二个花括号里面是新命令的内容,也就是我们使用这个新命令之后会进行什么操作,在这里我们定义这个指令的操作是输出加粗的文字“这是往年自定义的命令”,使用方法及效果如下
\begin{lstlisting}[language=TeX]
\MyCMD
\end{lstlisting}

\MyCMD

当然,第二个括号里面的内容也可以由大家自己设计,可以是任意命令的合法组合,从而实现各种操作。

此外,定义新命令的代码也可以这样写
\begin{lstlisting}[language=TeX]
\newcommand\MyCMD{\textbf{这是我们自定义的命令}}
\end{lstlisting}

这里两种写法没有任何实质性区别。

注意,\textbf{命令的命名只能是大小写字母,不能包括数字下划线等,否则会报错}。

这样,我们就可以把各种操作合并为一个简短的命令里面,从而大大提高我们的排版效率。
\section{带参数的命令}
但是,大家思考一下,上面的命令有什么问题呢?是不是过于死板了,只能执行既定的操作,除非你改变命令内部,否则一个命令只能执行这些操作,那我们在处理一些灵活任务的时候就会受到限制,所以我们就需要带参数的命令,就跟需要传入参数的函数一样,功能更强大。
首先我们定义一个带有一个参数的命令:
\begin{lstlisting}[language=TeX]
\newcommand{\MyCMD1}[1]{\textbf{#1}}
\end{lstlisting}
大家可以发现,这里定义新命令的时候,多了一个中括号,这个括号里面还有一个数字,这个数字就代表参数的个数,但是注意一下,\textbf{一个命令的参数个数最多是9个},然后我们展示一下用法与效果
\begin{lstlisting}[language=TeX]
\MyCMDone{hello}
\end{lstlisting}

\MyCMDone{hello}

在这里我们向命令传入了一个参数“hello”,然后根据我们定义的操作,将其加粗展示,在命令里面\#1就是代表第一个参数,大家也可以进行其他操作。

如果我们想同时处理多个参数,那么可以这样定义,比如我们定义一个三参数的命令:
\begin{lstlisting}[language=TeX]
\newcommand{\MyCMDtwo}[3]{(#1+#2)^#3}
$\MyCMDtwo{x}{y}{2}$
\end{lstlisting}

使用的时候,有几个参数就要用几个花括号括起来

$\MyCMDtwo{x}{y}{2}$

这样我们就实现了多参数命令的定义与使用。
\section{默认参数的命令}
除去以上用法,我们还可以定义带有默认参数的命令,如果我们不传入特定参数,那么就会使用默认的参数进行操作
\begin{lstlisting}[language=TeX]
\newcommand{\MyCMDthree}[3][x]{(#1+#2)^#3}
$\MyCMDthree{y}{2}$

$\MyCMDthree[z]{y}{2}$
\end{lstlisting}
$\MyCMDthree{y}{2}$

$\MyCMDthree[z]{y}{2}$

大家可以看到,在这个命令里面,当我们只传入的两个参数的时候,但是输出却包含三个参数,这就是第一个参数是默认参数导致的,在定义环节,我们在定义参数个数地方的后面使用方括号定义默认参数,如果我们想传入一个参数来代替默认的参数,那就在使用命令的时候,用方括号传入,同时要注意,\textbf{默认参数的个数只有一个,并且只能是第一个参数}。

如果我们想传入一个空参数,那么可以这样
\begin{lstlisting}[language=TeX]
$\MyCMDthree[]{y}{2}$
\end{lstlisting}

显示效果如下:

$\MyCMDthree[]{y}{2}$

可以看到本来应该显示x的地方消失了,这是因为我们传入了一个空参数,所以导致显示第一个参数的位置显示为空。
\begin{lstlisting}[language=TeX]

\end{lstlisting}