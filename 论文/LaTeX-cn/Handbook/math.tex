\chapter{数学公式}
\section{\LaTeX{}书写数学公式的好处}
各种涉及数学的专业的同学应该都看到过各种书籍、论文里面排版精美的数学公式,但是大家发现在Word里面书写数学公式总有各种各样的局限,尤其是在书写复杂公式和各种数学符号的时候,Word总是难以完成,但是LaTeX不存在这个问题,使用LaTeX可以排版出非常精美的公式,实际上大家所看到的论文中的公式,基本上都是由LaTeX进行排版的。
\section{书写第一个数学公式}
我们在文档里面输入以下内容
\begin{lstlisting}[language=TeX]
$$x^2$$
\end{lstlisting}

然后就可以看到对应位置有以下输出:
$$x^2$$

接下来我们来解析一下:首先,我们要使用\$\$将我们要显示的数学公式包起来,这样告诉编译器,我们要书写数学公式了,如果不这样,字母可以显示,但是很多数学符号就无法显示,还会报错,而使用\$\$将公式包起来,会让数学公式独占一行显示(也叫行间显示),如果不想单独一行显示,可以用\$将公式括起来
\begin{lstlisting}[language=TeX]
$x^2$
\end{lstlisting}

这样可以在行内显示,比如说这样:$x^2$。

然后x是字母,\^{}表示上标,\^{}跟在x后面,并且\^{}后面带有一个2表示上标的内容。

如果想让上标包含的内容更多,就需要将上标想显示的内容括起来
\begin{lstlisting}[language=TeX]
$$x^{1234}$$
$$x^1234$$
\end{lstlisting}
$$x^{1234}$$
$$x^1234$$

这是因为上标符号\^{}只会将后面第一个字符认定为上标内容,如果普通的多个字符(如ABC,123等),则只会将第一个字符变为上标,所以我们需要使用{}或者其他类似的命令,将一串字符变为一个整体,就可以让上标显示更多内容,同理,在其他数学符号里面也是如此。


