\chapter{QA}
\section{行列式}
\subsection{行列式的计算}
\subsubsection{具体型行列式}
\begin{itemize}
\item 一横型行列式 \label{一横型行列式}
\paragraph{例题} 一横形行列式$
\begin{bmatrix}
1-x & x & 0 \\
-1 & 1-x & x \\
0 & -1 & 1-x
\end{bmatrix}$=(\qquad)
\paragraph{解} 将其按第行展开, 得到一个相似的行列式, 可以得到一个递推公式.
\end{itemize}
\subsubsection{抽象型行列式}
\begin{itemize}
\item 目标行列式和矩阵的相互转换 \label{目标行列式和矩阵的相互转换}
\paragraph{例题} 设$\bm{\alpha_{1}},\bm{\alpha_{2}},...,\bm{\alpha_{n}}$是$n$维向量, $\bm{A}=[\bm{\alpha_{1}},\bm{\alpha_{2}},...,\bm{\alpha_{n}}],\bm{B}=[\bm{\alpha_{n}},\bm{\alpha_{1}},\bm{\alpha_{2}},...,\bm{\alpha_{n-1}}]$. 若$|\bm{A}|=1$, 则$|\bm{A}-\bm{B}|=(\quad)$\par
\item 与特征方程相结合 \label{与特征方程相结合}
\paragraph{例题} 设$\bm{A}$是$ 3 $阶方阵, 满足$ |3 \bm{A}+2 \bm{E}|=0, |\bm{A} - \bm{E}|=0, |3 \bm{E}-2 \bm{A}|=0 $, 则$ |\bm{A}|=(\qquad) $
\paragraph{解} 特征方程$ |\lambda \bm{E}-\bm{A}|=0 $, 可以根据上面的几个等式求出矩阵$ \bm{A} $的特征值, 根据特征值的性质可以知道矩阵的季和矩阵对应行列式的值
\end{itemize}
\subsection{余子式和代数余子式的线性组合的计算}\label{余子式和代数余子式的线性组合的计算}
\vspace{1em}
\paragraph{例题} 设$\bm{|A|}=
\begin{bmatrix}
2 & -1 & 2 & 3 \\
0 & -1 & -1 & 0 \\
2 & 3 & 4 & 5 \\
1 & 1 & 1 & 1
\end{bmatrix}
$, 则$A_{31}+A_{32}+A_{33}+M_{34}=
\begin{bmatrix}
2 & -1 & 2 & 3 \\
0 & -1 & -1 & 0 \\
1 & 1 & 1 & -1 \\
1 & 1 & 1 & 1
\end{bmatrix}
$
\section{矩阵}
\subsection{矩阵的基本运算}
\subsubsection{矩阵相乘时要注意矩阵的左右位置}\label{矩阵相乘时要注意矩阵的左右位置}
\paragraph{例题 1}~设$ \bm{E} $是$ n $阶单位矩阵, $ \bm{E}+\bm{A} $是$ n $阶可逆矩阵, 则下列关系式中不成立的是:
\begin{tasks}[label-offset={0.8em},label-format={\bfseries}](1)
\task $ (\bm{E}-\bm{A})(\bm{E}+\bm{A})^{2}=(\bm{E}+\bm{A})^{2}(\bm{E}-\bm{A}) $
\task $ (\bm{E}-\bm{A})(\bm{E}+\bm{A})^{T}=(\bm{E}+\bm{A})^{T}(\bm{E}-\bm{A}) $
\task $ (\bm{E}-\bm{A})(\bm{E}+\bm{A})^{-1}=(\bm{E}+\bm{A})^{-1}(\bm{E}-\bm{A}) $
\task $ (\bm{E}-\bm{A})(\bm{E}+\bm{A})^{*}=(\bm{E}+\bm{A})^{*}(\bm{E}-\bm{A}) $
\end{tasks} \par
\paragraph{解}~{} \noindent 题目中已经提示了$ \bm{E}+\bm{A} $是$ n $阶可逆矩阵, 所以下列成立:
\begin{equation*}
\begin{aligned}
& (\bm{E}+\bm{A})^{-1}(\bm{E}+\bm{A})=(\bm{E}+\bm{A})(\bm{E}+\bm{A})^{-1}\bm{E} \\
& (\bm{E}+\bm{A})^{*}(\bm{E}+\bm{A})=(\bm{E}+\bm{A})(\bm{E}+\bm{A})^{*}=|\bm{E}+\bm{A}|\bm{E}
\end{aligned}
\end{equation*}\par
可以将题目中的$ \bm{E}-\bm{A} $拆成$ 2\bm{E}-(\bm{E}+\bm{A}) $, 以$ b) $为例: 左式=$ (2 \bm{E}-(\bm{E}+\bm{A}))(\bm{E}+\bm{A})^{T}=2(\bm{E}+\bm{A})^{T}\bm{E}-(\bm{E}+\bm{A})(\bm{E}+\bm{A})^{T}$, 由于$ (\bm{E}+\bm{A})(\bm{E}+\bm{A})^{T}\neq (\bm{E}+\bm{A})(\bm{E}+\bm{A}) $, 故左式$ \neq $右式, 选$ b) $.
\paragraph{例题 2}~{已知$ \bm{E}_{32}(1)\bm{A}\bm{B}\bm{E}_{13}(-3)=
\begin{bmatrix}
1 & 0 & 0 \\
0 & -1 & 0 \\
0 & 1 & 2
\end{bmatrix}$, 求$ \bm{A}\bm{B} $}
\paragraph{解}~{注意只能左乘$ [\bm{E}_{32}(1)]^{-1} $, 右乘$ [\bm{E}_{13}(-3)]^{-1} $}
\section{向量}
\subsection{极大线性无关组及向量组秩的求法}
\paragraph{例题1}~{已知$ \bm{\alpha}_{1}, \bm{\alpha}_{2}, \bm{\alpha}_{3} $线性相关, 若$ \bm{\alpha}_{1}-3 \bm{\alpha_{3}}, a \bm{\alpha_{1}}+\bm{\alpha_{2}}+2 \bm{\alpha_{3}}, 2 \bm{\alpha_{1}}+3 \bm{\alpha_{2}}+\bm{\alpha_{3}} $亦线性无关, 则$ a $的取值} \label{初等变换不会改变矩阵的秩}
\paragraph{解}~{$ \bm{\alpha}_{1}-3 \bm{\alpha_{3}}, a \bm{\alpha_{1}}+\bm{\alpha_{2}}+2 \bm{\alpha_{3}}, 2 \bm{\alpha_{1}}+3 \bm{\alpha_{2}}+\bm{\alpha_{3}} $可以拆分成两个矩阵: 
\begin{equation*}
\begin{bmatrix}
\bm{\alpha}_{1} & \bm{\alpha}_{2} & \bm{\alpha}_{3}
\end{bmatrix}\times
\begin{bmatrix}
1 & a & 2 \\
0 & 1 & 3 \\
-3 & 2 & 1
\end{bmatrix}
\end{equation*}\par
由于$ \bm{\alpha}_{1}, \bm{\alpha}_{2}, \bm{\alpha}_{3} $线性无关, 根据一个矩阵乘以可逆矩阵之后其线性相关性不变, 可以得到这个矩阵是可逆的:
\begin{equation*}
\begin{bmatrix}
1 & a & 2 \\
0 & 1 & 3 \\
-3 & 2 & 1
\end{bmatrix}
\end{equation*}\par
故上述矩阵的行列式为$ 0 $, 进而求出$ a $的值为$ \frac{1}{9} $.